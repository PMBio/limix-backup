\documentclass{article}
\usepackage[utf8]{inputenc}
\usepackage{amssymb,amsfonts,amsmath}
\usepackage{graphicx}
\usepackage{caption}
\usepackage{subfig}
\usepackage{color}
\usepackage[affil-it]{authblk}
\usepackage{multirow}
\usepackage{fullpage}
\usepackage{booktabs}
\usepackage{pdfsync}

%\usepackage[nofiglist, notablist, nomarkers]{endfloat}
%%%%%%%%%%%%%%%%%%%%%%%%%%%%%%%%%%%%%%%%%%%%%%%%%%%%%%%%%%%%%%%%%%%%
%
%  File: utils.tex
%
%  Utilities for typesetting in latex
%
%  
%
%%%%%%%%%%%%%%%%%%%%%%%%%%%%%%%%%%%%%%%%%%%%%%%%%%%%%%%%%%%%%%%%%%%%

%integration
%\newcommand{\dint}{\mathrm{d}}
\newcommand{\rmd}{\mathrm{d}}

%Standard commands used throughout the thesis
\newcommand{\data}{\mathcal{D}}
\newcommand{\model}{\mathcal{H}}
\newcommand{\GPM}{\mathcal{H}_{\text{GP}}}
\newcommand{\datatest}{\mathcal{D}_{\text{test}}}
\newcommand{\pl}{\ensuremath{p_{\textnormal{L}}}}
\newcommand{\TK}{\ensuremath{\btheta_{\textnormal{K}}}}
\newcommand{\hTL}{\ensuremath{\hat{\btheta}_{\textnormal{L}}}}
\newcommand{\hTK}{\ensuremath{\hat{\btheta}_{\textnormal{K}}}}
\newcommand{\TL}{\ensuremath{\btheta_{\textnormal{L}}}}
\newcommand{\TT}{\ensuremath{\btheta}}
\newcommand{\x}{\ensuremath{\bfx}}
\newcommand{\X}{\ensuremath{\bfX}}
%GP symboles
\newcommand{\HGP}{\ensuremath{\mathcal{H}_{\textnormal{GP}}}}

\newcommand{\argmax}{\operatornamewithlimits{argmax}}
%commonly used symbols
\newcommand{\DC}{\ensuremath{^{\circ}\mathrm{C}}}


%MathBold-Font
\newcommand{\mbf}[1]{{\ensuremath{\mathbf{#1}}}}

%Real Numbers, C etc.
\newcommand{\R}{{\sf R\hspace*{-0.9ex}\rule{0.15ex}%
    {1.5ex}\hspace*{0.9ex}}}
\newcommand{\N}{{\sf N\hspace*{-1.0ex}\rule{0.15ex}%
    {1.3ex}\hspace*{1.0ex}}}
\newcommand{\Q}{{\sf Q\hspace*{-1.1ex}\rule{0.15ex}%
    {1.5ex}\hspace*{1.1ex}}}
\newcommand{\C}{{\sf C\hspace*{-0.9ex}\rule{0.15ex}%
    {1.3ex}\hspace*{0.9ex}}}

\newcommand{\be}{\begin{equation}}
\newcommand{\ee}{\end{equation}}
\newcommand{\bea}{\begin{eqnarray}}
\newcommand{\eea}{\end{eqnarray}}
\newcommand{\beas}{\begin{eqnarray*}}
\newcommand{\eeas}{\end{eqnarray*}}

\newcommand{\const}{{\rm const.}}
\newcommand{\matlab}{${\rm Matlab}^{\Pisymbol{psy}{226}}$}


% A comment
\newcommand{\comment}[1]{}

\newcommand{\TODO}[1]{{\color{red}\fbox{TODO} #1}}
\newcommand{\CHANGE}[1]{{\color{blue} #1}}

% Distributions
\newcommand{\normal}[2]{\mathcal{N}\left(#1 \,\middle | \, #2\right)}
\newcommand{\dgamma}[2]{\Gamma\left(#1 \given #2\right)}

%true,false
\newcommand{\true}{\text{true}}
\newcommand{\false}{\text{false}}

% calH, calD
\newcommand{\calH}{\mathcal{H}}
\newcommand{\calD}{\mathcal{D}}

%
\newcommand{\indep}{\bot \hspace{-0.6em} \bot}
\newcommand{\arrow}{\rightarrow}
\newcommand{\given}{\,|\,}
\newcommand{\twolines}{\,||\,}
\newcommand{\narroweq}{\!\!=\!\!}

% Parents and children
\newcommand{\pa}[1]{{\rm pa_\mathit{#1}}}
\newcommand{\cp}[2]{{\rm cp_\mathit{#1}^{(\mathit{#2})}}}
\newcommand{\ch}[1]{{\rm ch_\mathit{#1}}}

% Neighbours
\newcommand{\neigh}[1]{{\rm ne_\mathit{#1}}}

% KL divergence
%\newcommand{\KL}{{\rm KL}}
%\newcommand{\KLVB}{{\rm KL}_{\text{VB}}
%\newcommand{\KLEP}{{\rm KL}_{\text{EP}}

% Entropy
\newcommand{\entropy}{{\mathbb{H}}}


\newcommand{\cip}{\mbox{$\perp\!\!\!\perp$}}
\newcommand{\condindep}[3]{#1~\cip~#2~|~#3}
\newcommand{\nocondindep}[3]{#1~\mbox{$\not\!\perp\!\!\!\perp$}~#2~|~#3}
\newcommand{\dir}[2]{{\rm Dir}(#1|#2)}

\newcommand{\bDelta}{\mbox{\boldmath $\Delta$}}
\newcommand{\bbeta}{\mbox{\boldmath $\beta$}}
\newcommand{\bmu}{\mbox{\boldmath $\mu$}}
\newcommand{\bnu}{\mbox{\boldmath $\nu$}}
\newcommand{\balpha}{\mbox{\boldmath $\alpha$}}
\newcommand{\bepsilon}{\mbox{\boldmath $\epsilon$}}
\newcommand{\bgamma}{\mbox{\boldmath $\gamma$}}
\newcommand{\bsigma}{\mbox{\boldmath $\sigma$}}
\newcommand{\bSigma}{\mbox{\boldmath $\Sigma$}}
\newcommand{\btau}{\mbox{\boldmath $\tau$}}
\newcommand{\blambda}{\mbox{\boldmath $\lambda$}}
\newcommand{\bLambda}{\mbox{\boldmath $\Lambda$}}
\newcommand{\bpi}{\mbox{\boldmath $\pi$}}
\newcommand{\bpsi}{\mbox{\boldmath $\psi$}}
\newcommand{\bchi}{\mbox{\boldmath $\chi$}}
\newcommand{\bxi}{\mbox{\boldmath $\xi$}}
\newcommand{\bPsi}{\mbox{\boldmath $\Psi$}}
\newcommand{\bphi}{\mbox{\boldmath $\phi$}}
\newcommand{\bPhi}{\mbox{\boldmath $\Phi$}}

\newcommand{\btheta}{\mbox{\boldmath $\theta$}}
\newcommand{\bdelta}{\mbox{\boldmath $\delta$}}
\newcommand{\bTheta}{\mbox{\boldmath $\Theta$}}
\newcommand{\bOmega}{\mbox{\boldmath $\Omega$}}

\newcommand{\Bmath}[1]{\mbox{\boldmath $#1$}}

\newcommand{\fastfig}[4]{
\begin{center}
\begin{figure}[htb!]
\centerline{\epsfig{figure=#1,width=#2}}
\caption[short]{#3}
\label{#4}
\end{figure}
\end{center}
}

\newcommand{\unit}{{\bf I}}
\newcommand{\boldzero}{{\bf 0}}

\newcommand{\bfa}{{\bf a}}
\newcommand{\bfb}{{\bf b}}
\newcommand{\bfc}{{\bf c}}
\newcommand{\bfd}{{\bf d}}
\newcommand{\bfe}{{\bf e}}
\newcommand{\bff}{{\bf f}}
\newcommand{\bfg}{{\bf g}}
\newcommand{\bfh}{{\bf h}}
\newcommand{\bfi}{{\bf i}}
\newcommand{\bfk}{{\bf k}}
\newcommand{\bfl}{{\bf l}}
\newcommand{\bfm}{{\bf m}}
\newcommand{\bfp}{{\bf p}}
\newcommand{\bfr}{{\bf r}}
\newcommand{\bfs}{{\bf s}}
\newcommand{\bft}{{\bf t}}
\newcommand{\bfu}{{\bf u}}
\newcommand{\bfv}{{\bf v}}
\newcommand{\bfw}{{\bf w}}
\newcommand{\bfx}{{\bf x}}
\newcommand{\bfy}{{\bf y}}
\newcommand{\bfz}{{\bf z}}

\newcommand{\bfA}{{\bf A}}
\newcommand{\bfB}{{\bf B}}
\newcommand{\bfC}{{\bf C}}
\newcommand{\bfD}{{\bf D}}
\newcommand{\bfG}{{\bf G}}
\newcommand{\bfH}{{\bf H}}
\newcommand{\bfI}{{\bf I}}
\newcommand{\bfJ}{{\bf J}}
\newcommand{\bfK}{{\bf K}}
\newcommand{\bfL}{{\bf L}}
\newcommand{\bfM}{{\bf M}}
\newcommand{\bfQ}{{\bf Q}}
\newcommand{\bfR}{{\bf R}}
\newcommand{\bfS}{{\bf S}}
\newcommand{\bfF}{{\bf F}}
\newcommand{\bfT}{{\bf T}}
\newcommand{\bfU}{{\bf U}}
\newcommand{\bfV}{{\bf V}}
\newcommand{\bfW}{{\bf W}}
\newcommand{\bfX}{{\bf X}}
\newcommand{\bfY}{{\bf Y}}
\newcommand{\bfZ}{{\bf Z}}
\newcommand{\llangle}{{\langle \hspace{-0.7mm} \langle}}
\newcommand{\rrangle}{{\rangle \hspace{-0.7mm} \rangle}}
\newcommand{\define}{\stackrel{\mathrm{def}}{=}}

\newcommand{\la}{\langle}
\newcommand{\ra}{\rangle}
\newcommand{\La}{\left\langle}
\newcommand{\Ra}{\right\rangle}
\newcommand{\EXP}[1]{\left\langle #1 \right\rangle}
\newcommand{\vectwo}[2]{\left[\begin{array}{c} #1 \\ #2 \end{array}\right]}
\newcommand{\vecn}[1]{\left[\begin{array}{c} #1 \end{array}\right]}
\newcommand{\half}{{\scriptstyle \frac{1}{2}}}
\newcommand{\col}{\mathrm{vec}}
\newcommand{\colI}{\mathrm{vecI}}
\newcommand{\fl}[1]{\mathrm{flatten}\left(#1\right)}
\newcommand{\fli}[1]{\mathrm{flattenI}\left(#1\right)}
\newcommand{\trans}[1]{{#1}^{\ensuremath{\mathsf{T}}}}
\newcommand{\T}{{\rm T}}
\newcommand{\diag}{{\rm diag}}
\newcommand{\Tr}{\mbox{Tr}}
\newcommand{\diff}[1]{{\,d#1}}
\newcommand{\vgraph}[1]{
  \newpage
  \begin{center}
  {\large \bf #1}
  \end{center}
  \vspace{2mm}
}
\newcommand{\high}[1]{\textcolor{blue}{\emph{#1}}}
\newcommand{\cut}[1]{}
\newcommand{\citeasnoun}[1]{\citeN{#1}}
\newcommand{\citemulti}[2]{(#1, \citeyearNP{#2})}
\newcommand{\citemultiN}[2]{#1 (\citeyearNP{#2})}
\newcommand{\Sum}{{\displaystyle \sum}}
%\newcommand{\sumint2}{\operatorname*{\sum \!\!\!\!\!\!\!\!\!\!\!\!
%\int}}
\newcommand{\msumint}{\operatorname*{\sum \!\!\!\!\!\!\!\! \int}}


%%%%%%%%%%%%%%%%%%%%%%%%%%%%%%%%%%%%%%%%%%%%%%%%%%%%%%%%%%%%%%%%%%%%

%\title{Probabilistic learning of confounding factors in genetical
%genomics studies}
\title{Derivations and mathematical details of GPmix}
\author[1]{Christoph Lippert, Oliver Stegle}

\affil[1]{Department Empirical Inference,
Max Planck Institutes T\"ubingen, Germany}
\date{}

\captionsetup[subfloat]{listofformat=parens}
\newcommand{\OLI}[1]{{\color{blue}\fbox{OLI} #1}}
\makeatletter
\newcommand{\rmnum}[1]{\romannumeral #1}
\newcommand{\Rmnum}[1]{\expandafter\@slowromancap\romannumeral #1@}
\makeatother

\newcommand{\fix}{\marginpar{FIX}}
\newcommand{\new}{\marginpar{NEW}}

%\nipsfinalcopy % Uncomment for camera-ready version
%\newcommand{\B}[1]{\bm{#1}} Christoph:removed this to make things consistent
\newcommand{\B}[1]{{\bf{#1}}}
\newcommand{\Exp}{\mathbb{E}}
\newcommand{\norma}[1]{\mathcal{N}\left(#1\right)}
\newcommand{\eref}[1]{(\ref{#1})}
\renewcommand{\R}{\mathbb{R}}
\newcommand\norm[1]{\left\Vert {#1} \right\Vert}
\newcommand\rank{\mathrm{rank}}
\newcommand\tr{\mathrm{Tr}}
\newcommand\Normal[3]{\normal{#1}{{#2}\;;\;{#3}}}
\newcommand\ve[1]{\text{vec}\left(#1\right)}
\newcommand\Real{\mathbb{R}}
\newcommand{\Ykron}{\B{U}_R^\T\B{Y}\B{U}_C}
\newcommand{\XWAkron}[1]{\B{U}_R^\T\B{X}_{#1}\B{W}_{#1}\B{A}_{#1}\B{U}_C}
\newcommand{\XWAkronT}[1]{\B{U}_C^\T\B{A}_{#1}^\T\B{W}_{#1}^{\T}\B{X}_{#1}^{\T}\B{U}_R}
\begin{document}
\maketitle


\section{Some definitions}
\begin{itemize}
\item $\B{A}=\B{U}\B{S}\B{U}^{\T}$ is the eigenvalue decomposition of
  the symmetric $D$-by-$D$ matrix $\B{A}$, where $\B{U}$ is an
  $D$-by-$D$ orthonormal matrix, holding the $D$ eigenvectors of
  $\B{A}$ and $\B{S}$ is an $D$-by-$D$ diagonal matrix, holding the
  corresponding eigenvalues of $\B{A}$ as diagonal entries. 
\item $\B{A}\odot\B{B}$ is the pointwise or Hadamard product of $\B{A}$ and $\B{B}$.
\item $\B{A}\otimes\B{B}$ is the Kronecker product of $\B{A}$ and $\B{B}$.
\item $\B{Y}\in\mathbb{R}^{N\times G}$ is the matrix holding all samples, having $N$ rows and $G$ columns.
\end{itemize}


\section{Kronecker testing strategies and models}
$\B{A}_j\in\Real^{C\times M_j}$ is a matrix, that replicates the
$j$-th fixed effects matrix $\B{X}_j\in\Real^{R\times D_j}$. $\B{A}_j^\T$
typically would be a binary matrix, but could in principle be
anything. Using different versions of $\B{A}_j^\T$ corresponds to chosing
a testing strategy. For example, when $\B{A}_j^\T$ is the $C\times C$
Identity matrix, then one would fit an independent weight to every
column of $\B{Y}$, when $\B{A}^\T_j$ is a row-vector of ones, then one
would fit a single joint weight to all columns of $\B{Y}$.
\begin{equation}
\Normal{\ve{\B{Y}}}{\sum_j\B{A}^{\T}_j\otimes\B{X}_j\ve{\B{W}_j}}{\B{C}\otimes\B{R}
+ \sigma^2\B{I}}
\end{equation}
As long as $D_j\leq R$, $M_j\leq C$, the rank of $\B{X}_j$ is $D_j$
and the rank of $\B{A}_j$ is $M_j$, and the rank of
$\left[\B{X}_1...\B{X}_J\right]$ is $\sum_{j=1}^JM_j$ the number of
degrees of freedom of a single $\B{W}_j$ is $D_j \cdot M_j$
(sufficient condition).
\section{Efficient computation of tensor GP models}
\begin{equation}
\log\Normal{\ve{\B{Y}}}{\sum_{j=1}^J\B{A}^{\T}_j\otimes\B{X}_j\ve{\B{W}_j}}{\B{C}\otimes\B{R}
+ \sigma^2\B{I}}
\end{equation}
Apply the vec-trick to the mean term:
\begin{equation}
\log\Normal{\ve{\B{Y}}}{\ve{\sum_{j=1}^J\B{X}_j\B{W}_j\B{A}_j}}{\B{C}\otimes\B{R}
+ \sigma^2\B{I}}
\end{equation}
\begin{equation}
-\frac{C\cdot R}{2}\log(2\pi)-\frac{1}{2}\log|\B{C}\otimes\B{R} +
\sigma^2\B{I}|
-\frac{1}{2}\ve{\B{Y} -\sum_{j=1}^J\B{X}_j\B{W}_j\B{A}_j}^\T\left(\B{C}\otimes\B{R} +
\sigma^2\B{I}\right)^{-1}\ve{\B{Y} -\sum_{j=1}^J\B{X}_j\B{W}_j\B{A}_j}
\end{equation}
\subsection{Derivative of the squared form wrt. $\B{W}$}
\begin{equation}
\frac{\partial}{\partial [\B{W}_k]_{ab}}\left(-\frac{1}{2}\ve{\B{Y} -\sum_{j=1}^J\B{X}_j\B{W}_j\B{A}_j}^\T\left(\B{C}\otimes\B{R} +
\sigma^2\B{I}\right)^{-1}\ve{\B{Y} -\sum_{j=1}^J\B{X}_j\B{W}_j\B{A}_j}\right)
\end{equation}
We define matrix $\B{D}\in\Real^{R\times C}$, such that $\ve{\B{D}} = \diag\left(\B{S}_C\otimes\B{S}_R +
\sigma^2\B{I}\right)^{-1}$ and rotate the data:
\begin{equation}
\frac{\partial}{\partial [\B{W}_k]_{ab}}
-\frac{1}{2}\tr\left(\left(\Ykron
    -\sum_{j=1}^J\XWAkron{j}\right)^\T\left(\left(\Ykron
      -\sum_{j=1}^J\XWAkron{j}\right)\odot \B{D}\right)\right)
\end{equation}
\begin{eqnarray}\nonumber
\frac{\partial}{\partial [\B{W}_k]_{ab}}
-\frac{1}{2}\tr\left(\Ykron^\T (\Ykron \odot \B{D})\right)
    +\tr\left( \sum_{j=1}^J\XWAkronT{j}(\Ykron \odot \B{D})\right) \\
-\tr\left( \sum_{j=1}^J\sum_{i=j+1}^J\XWAkronT{j}(\XWAkron{i} \odot
  \B{D}) \right) 
-\frac{1}{2}\tr\left(\sum_{j=1}^J\XWAkronT{j}(\XWAkron{j} \odot \B{D}) \right)
\end{eqnarray}
Leaving out all terms without $\B{W}_k$:
\begin{eqnarray}\nonumber
\frac{\partial}{\partial [\B{W}_k]_{ab}}
    \tr\left( \XWAkronT{k}(\Ykron \odot \B{D})\right) \\
-\tr\left( \sum_{j\neq k}\XWAkronT{k}(\XWAkron{j} \odot
  \B{D}) \right) 
-\frac{1}{2}\tr\left(\XWAkronT{k}(\XWAkron{k} \odot \B{D}) \right)
\end{eqnarray}

\begin{eqnarray}\nonumber
    \tr\left( {\B{U}_C^\T[\B{A}_{k}]_{b:}^{\T}[\B{X}_{k}]_{:a}^{\T}\B{U}_R} (\Ykron \odot \B{D})\right) \\
-\tr\left( \sum_{j\neq k}{\B{U}_C^\T[\B{A}_{k}]_{b:}^{\T}[\B{X}_{k}]_{:a}^{\T}\B{U}_R} (\XWAkron{j} \odot
  \B{D}) \right) 
-\tr\left({\B{U}_C^\T[\B{A}_{k}]_{b:}^{\T}[\B{X}_{k}]_{:a}^{\T}\B{U}_R} (\XWAkron{k} \odot \B{D}) \right)
\end{eqnarray}
Moving around the terms in the trace and combining the last two traces
into one sum:
\begin{eqnarray}\nonumber
    \tr\left( [\B{X}_{k}]_{:a}^{\T}\B{U}_R (\Ykron \odot \B{D}) \B{U}_C^\T[\B{A}_{k}]_{b:}^{\T}\right) \\
-\tr\left( \sum_{j=1}^J{[\B{X}_{k}]_{:a}^{\T}\B{U}_R} (\XWAkron{j} \odot
  \B{D}) \B{U}_C^\T[\B{A}_{k}]_{b:}^{\T}\right) 
\end{eqnarray}
Observing that this is a scalar, we can eliminate the trace and re-arrange:
\begin{eqnarray}
    [\B{X}_{k}]_{:a}^{\T}\B{U}_R ((\Ykron-\sum_{j=1}^J \XWAkron{j}) \odot \B{D}) \B{U}_C^\T[\B{A}_{k}]_{b:}^{\T} \\
\end{eqnarray}
Stacking together these terms for all $a$ and $b$, the gradient becomes
\begin{eqnarray}
    \B{X}_{k}^{\T}\B{U}_R ((\Ykron-\sum_{j=1}^J \XWAkron{j}) \odot \B{D}) \B{U}_C^\T\B{A}_{k}^{\T} \\
\end{eqnarray}

\newpage
\end{document}
